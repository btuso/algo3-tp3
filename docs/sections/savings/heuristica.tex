\subsection{Heuristica de Savings}
\subsubsection{El algoritmo}
Si consideramos una solución inicial al problema de CVRP como un conjunto de $n$ camiones (siendo $n$ la cantidad de clientes a visitar) y suponiendo que salen del depósito, visitan cada uno un cliente distinto y vuelven al punto de partida entonces el trayecto de un camión que visite al nodo $i$ será $2*d(D,i)$. En consecuencia, la suma de todas las distancias recorridas por estos será $\sum_{1}^{n} 2*d(D, i)$.\\
Ahora si en vez de eso usamos un camión para visitar a dos clientes $i$ y $j$ (o que es lo mismo, dos nodos $i$ y $j$):
$$ distancia\_recorrida = d(D, i) + d(i,j) +d(D, j)     $$
Por lo tanto lo que ahorramos al ir a ambos en el mismo recorrido es:\\
$$ savings(i,j) = 2*d(D, i) + 2* d(D,j) -  (d(D, i) + d(i,j) +d(D, j))     $$
$$ savings(i,j) = d(D, i) + d(D,j) -   d(i,j)    $$
La idea es crear recorridos para los camiones que usen los savings más altos posibles. Para ello, utilizamos el algoritmo de Clark-Wright:

\begin{description}
	\item[Paso 1.] Para todo par $(i,j)$ se calcula $savings(i,j)$. Las formas de elegir dos nodos entre $n$ es ${n}\choose{2}$$ = n(n-1)/2$.
	\item[Paso 2.] Ordenar $savings$ de mayor a menor y recorrerlos de esta forma.
	\item[Paso 3.] Por cada $saving$ que incluya los nodos $i$ y $j$, decidiremos si lo utilizaremos o no en nuestra solución.
	\item[Paso 4.] Repetimos paso 3. hasta que no haya $saving$ que incluir.
\end{description}
Para el paso 4, tendremos los siguientes criterios de inclusión:
\begin{itemize}
	\item \textbf{Ninguno de los nodos fue visitado aun:} crearemos uno cuyo recorrido sea visitar ambos clientes, siempre y cuando la capacidad del camión sea mayor o igual que las demandas de ambos juntas.
	\item \textbf{Un nodo no fue visitado aun pero el otro sí:} siempre y cuando el nodo ya visitado no sea interno (es decir, que no sea el primero o último cliente de la ruta) y la capacidad actual de su respectivo camión sea mayor o igual que la demanda del nodo no visitado.
	\item \textbf{Ambos nodos pertenecen a rutas distintas:} si ninguno es interno y la distancia recorrida de ambas rutas es menor o igual que la capacidad de uno de los camiones, uniremos las dos rutas en una.
\end{itemize}

\subsubsection{Pseudo-código}
En primer lugar, aclararemos ciertos aspectos de los tipos de datos utilizados para implementar la solución. Se tiene un tipo de dato \textbf{Camión} el cual almacena su stock disponible, si es válido, cual es el último cliente y dos vectores de tamaño $n$ que indican para todo $i$ el nodo que precede y el siguiente a $i$ (el siguiente del último nodo es nulo y el predecesor del primero también). Estos últimos vectores nos serviran para mergear las rutas según el paso 4 y para reconstruir los caminos. \\
Además, tenemos el tipo \textbf{Saving} el cual almacenará los dos nodos involucrados y el respectivo saving según el paso 1.

\begin{algorithm}[H]
	\caption{\Comment $\mathcal{O}(n^{3})$}
	\begin{algorithmic}[1]
		\Function{resolverCVRP}{Punto $deposito$, Conjunto de puntos $puntos$, Entero $capacidad$}
		\State $n \gets |puntos|$
		\State \textbf{Matriz de Reales} $distancias\mathcal{}(n^{2}, 0)$
		\State \textbf{Vector de Reales} $distancia\_a\_deposito\mathcal{}(n, 0)$
		\State \textbf{Vector de Enteros} $en\_que\_camion\mathcal{}(n, ninguno)$
		\State \textbf{Vector de Camiones} $camiones \gets \varnothing$
		\State \textbf{Vector de Savings} $savings \gets \varnothing$
		\Statex
		\State \textbf{calcularDistancias}$(distancias, distancia\_a\_deposito, puntos, deposito)$ 
		\State \textbf{calcularSavings}$(distancias, distancia\_a\_deposito, puntos, savings)$ 
		\State \textbf{ordenarDecreciente}$(savings)$
		\Statex
		\While{s=(i,j) en $savings$ sea no negativo}
		\If{los nodos de s no son visitados por ningun camión}
		\State \textbf{Entero } $demanda \gets  puntos[i].demand + puntos[j].demand$
		\If{$capacidad >= demanda$}
		\State \textbf{camionNuevo}$(camiones, en\_que\_camion, i, j, demanda)$
		\EndIf
		\EndIf
		\If{el nodo i es visitado por un camion y el nodo j no}
		\If{\textbf{puedoAgregarlo}$(camiones[i], i, puntos[j].demand)$} 
		\State \textbf{visitarCliente}$(camiones[i], en\_que\_camion, i, j, puntos[j].demand)$
		\EndIf
		\EndIf
		\If{el nodo j es visitado por un camion y el nodo i no}
		\If{\textbf{puedoAgregarlo}$(camiones[j], j, puntos[i].demand)$}
		\State \textbf{visitarCliente}$(camiones[j], en\_que\_camion, j, i, puntos[i].demand)$
		\EndIf
		\EndIf
		\If{ambos son visitados por camiones}
		\If {\textbf{puedoUnirRutas}$(camiones[i],camiones[j],i,j)$}
		\State {\textbf{unirRutas}$(camiones, en\_que\_camion,i,j)$}
		\EndIf
		\EndIf
		\EndWhile
		\Statex
		\If{el nodo $i$ no esta en ningun camion}
		\State $camiones$\textbf{.agregarAtras}$(Camion(capacidad, n, i, puntos[i].demand))$
		\EndIf
		\State \textbf{armarCamiones}$(camiones, puntos, distancias, distancia\_a\_deposito)$
		\State \Return $camiones$
		\EndFunction
	\end{algorithmic}
\end{algorithm}

\paragraph{Inicialización y Savings}

\begin{itemize}
	\item Calcularemos la distancia euclidea entre cada par de puntos $(i,j)$ y la almacenaremos en $distancias[i][j]$ mediante \textbf{calcularDistancias}. En la misma función, también almacenaremos en $distancia\_a\_deposito$ la distancia de cada nodo a este según .
	\item Para cada par de nodos $(i,j)$, a partir de las distancias calculadas calcularemos el $saving$ y lo agregaremos a $savings$.  
	\item Ordenar de forma creciente y luego iterar al revés es lo mismo que ordenar de forma decreciente a fines funcionales. Usaremos un sort de la std.
\end{itemize}

\paragraph{Ciclo}\hspace{0pt} \\
\\
De nuevo, recorreremos el ciclo del mayor saving al menor. Para cada uno, habrá cuatro casos que nos percaten:
\begin{description}
	\item[Caso 1.] Ni $i$ ni $j$ son visitados, si la demanda entre ambos no supera $capacidad$ entonces hay que crear un camión nuevo que los visite.  \textbf{camiónNuevo} crea un camión mediante  \ref{nuevo-truck} del tipo \textbf{Camión} y actualiza $en\_que\_camion$ para $i$ y $j$ con el recién creado.
	\item[Caso 2 y 3.] Exactamente uno de los dos nodos ya fue visitado. Se llama a la función \textbf{puedoAgregarlo} pero alterando los parámetros. Esta chequea que en el camión haya espacio para la demanda del no visitado y que el existente no sea nodo interno (no puede tener un siguiente y un predecesor al mismo tiempo). Si todo sale bien, agrega el nodo no visitado al camión del visitado segun \ref{visitar-nodo} y actualiza $en\_que\_camion[nodo\_no\_visitado]$.
	\item[Caso 4.] Los nodos están en distintos camiones. La función \textbf{puedoUnirRutas}, al igual que \textbf{puedoAgregarlo} chequea que la suma de las distancias de las rutas de los dos camiones involucrados no supere la capacidad de un camión (que es la misma para todos) y que ninguno de los nodos sea interno. En caso de que sea posible la unión, mergea según \textbf{Unión de rutas}.
\end{description}

\begin{algorithm}[H]
	\caption{\Comment $\mathcal{O}(n)$}
	\label{nuevo-truck}
	\begin{algorithmic}[1]
		\Function{Camión}{Entero $total\_capacity$, Entero $n$, Entero $i$, Entero $j$, Entero $demanda$}
		\State $predecesores \gets$ Vector de Enteros(n, ninguno)
		\State $siguientes \gets$ Vector de Enteros(n, ninguno)
		\State $predecesores[j] \gets i$
		\State $siguientes[i] \gets j$
		\State $cliente\_final \gets j$	
		\State $stock\_left \gets total\_capacity - demanda$ 
		\State $es\_valido \gets  true$ 
		\EndFunction
	\end{algorithmic}
\end{algorithm}

\begin{algorithm}[H]
	\caption{\Comment $\mathcal{O}(1)$}
	\label{visitar-nodo}
	\begin{algorithmic}[1]
		\Function{Visitar}{Entero $existente$, Entero $nuevo$, Entero $demanda$}
		\If{predecesores[existente]=ninguno}
		\State $predecesores[existente] \gets nuevo$
		\State $siguientes[nuevo] \gets existente$
		\Else 
		\State $predecesores[nuevo] \gets existente$
		\State $siguientes[existente] \gets nuevo$
		\State $cliente\_final \gets nuevo$
		\EndIf
		\State $stock\_left \gets stock\_left-demanda$
		\EndFunction
	\end{algorithmic}
\end{algorithm}
\paragraph{Unión de rutas} \hspace{0pt} \\
\\
Para un par de nodos $(i,j)$, elegimos unir ambas rutas en $camion[i]$ (sea $predecesores\_B$ los predecesores del $camion[j]$), obteniendo cuatro casos:
\begin{enumerate}
	\item $predecesores[i]=ninguno$  y $predecesores\_B[j] = ninguno$ entonces conecto $i$ y $j$ (predecesor de $i$ es $j$ y siguiente de $j$ es $i$). Iremos recorriendo $siguientes\_B$ desde $j$ y consistentemente conectar esos nodos con lo $predecesores$ y $siguientes$. 
	\item $predecesores[i]=ninguno$  y $predecesores\_B[j] \neq ninguno$ entonces conecto $i$ y $j$ (predecesor de $i$ es $j$ y siguiente de $j$ es $i$). Ahora no hay que invertir el orden, sólo copiamos desde el camión de $j$ todos los nodos en $predecesores\_B$ y $siguientes\_B$ que no sean nulls a los del camión $i$.
	\item $predecesores[i] \neq ninguno$ y $predecesores\_B[j] = ninguno$ entonces  $i$ y $j$ (predecesor de $j$ es $i$ y siguiente de $i$ es $j$). Iremos recorriendo $predecesores\_B$ desde $j$ y consistentemente conectar esos nodos con lo $predecesores$ y $siguientes$. El nuevo cliente final será el primero del $camion[j]$.
	\item $predecesores[i] \neq ninguno$ y $predecesores\_B[j] \neq ninguno$ entonces  $i$ y $j$ (predecesor de $j$ es $i$ y siguiente de $i$ es $j$). Ahora no hay que invertir el orden, sólo copiamos desde el camión de $j$ todos los nodos en $predecesores\_B$ y $siguientes\_B$ que no sean nulls al los del camión $i$. El nuevo cliente final será el cliente final del $camion[j]$.
\end{enumerate}
\paragraph{Armado de rutas} \hspace{0pt} \\
\\
Terminado el ciclo cuando los savings comienzan a ser negativos, debemos chequear si faltó visitar a algún cliente. Para ello, iteramos sobre $en\_que\_camion$ y cuando encontremos que alguno está en $ninguno$, creamos un $Truck$ que visite sólo a ese nodo de forma similar a \ref{nuevo-truck} mediante \ref{visitar-un-nodo-nuevo-truck}.
\begin{algorithm}[H]
	\caption{\Comment $\mathcal{O}(n)$}
	\label{visitar-un-nodo-nuevo-truck}
	\begin{algorithmic}[1]
		\Function{Camión}{Entero $total\_capacity$, Entero $n$, Entero $i$, Entero $demanda$}
		\State $predecesores \gets$ Vector de Enteros(n, ninguno)
		\State $siguientes \gets$ Vector de Enteros(n, ninguno)
		\State $cliente\_final \gets i$	
		\State $stock\_left \gets total\_capacity - demanda$ 
		\State $es\_valido \gets  true$ 
		\EndFunction
	\end{algorithmic}
\end{algorithm}
Por último,  $armarCamiones$ llamará por cada camión válido a la función auxiliar $armarRuta$. Esta navegará por el arreglo de predecesores de cada camión a partir del cliente final y mientras no llegué al primero (cuyo predecesor es null). Agregará cada punto correspondiente a la componente $routes$ de cada camión, generando el camino. 
\begin{algorithm}[H]
	\caption{\Comment $\mathcal{O}(n)$}
	\label{armar-ruta}
	\begin{algorithmic}[1]
		\Function{armarRuta}{Camion $camion$, Vector de Puntos $puntos$, Matriz de Reales $distancias$, Vector de Reales $distancia\_a\_deposito$}
		\State \textbf{Entero} $cliente \gets camion.cliente\_final$
		\State $camion.routes$\textbf{.agregarAtras}$(puntos[cliente])$
		\While{$camion.predecesores[cliente] \neq ninguno$}
		\State $cliente \gets camion.predecesores[cliente]$
		\State $camion.routes$\textbf{.agregarAtras}$(puntos[cliente])$
		\EndWhile
		\EndFunction
	\end{algorithmic}
\end{algorithm}
\subsubsection{Análisis de complejidad}
\paragraph{Complejidad espacial}\hspace{0pt} \\
\\
Contamos con varios vectores de tamaño $n$ y la matriz de $distancias$ de tamaño $n^{2}$. Además, por cada camión que se cree se tienen dos vectores de tamaño $n$, $predecesores$ y $siguientes$. Como la cantidad de camiones está acotada por $n$, la complejidad espacial pertenece a $\mathcal{O}(n^{2})$.
\paragraph{Complejidad temporal}
\begin{itemize}
	\item Inicialización de matriz en $\mathcal{O}(n^{2})$ y de vectores en $\mathcal{O}(n)$.
	\item Recorrer los $n(n-1)/2$ savings pertenece a $\mathcal{O}(n^{2})$.
	\item Para cada uno de ellos, se efectua una de las tres opciones:
	\subitem 1. Se crea un camión nuevo. Como se inicializan $predecesores$ y $siguientes$ de tamaño $n$, pertenece a $\mathcal{O}(n)$.
	\subitem 2. Uno de los nodos será visitado por el camión del otro. Se efectuan simples asignaciones por lo que pertenece a $\mathcal{O}(1)$.
	\subitem 3. Se mergean rutas. Se actualizara $en\_que\_camion$ para todos los nodos que visitaba el camión que se invalidará, mientras que se actualizarán los $predecesores$ y $siguientes$ del camión con las rutas unidas. Pertenece a $\mathcal{O}(n)$.
	\item Los chequeos para saber si se pueden unir/mergear son simples comparaciones: $\mathcal{O}(1)$.
	\item Se recorre $en\_que\_camion$ en $\mathcal{O}(n)$ y se crean los camiones faltantes en $\mathcal{O}(n)$.
	\item El armado de rutas recorre los camiones válidos y agrega todos los nodos a sus respectivas rutas. Recorre en a lo sumo $\mathcal{O}(n)$ y el armado total es de $\mathcal{O}(n)$ ya que hay $n$ nodos y cada uno está en un solo camión.
\end{itemize}
En conclusión, sabemos que el algoritmo pertenece a $\mathcal{O}(n^{3})$.