\subsection{Heurística Constructiva Golosa}
Utilizamos un enfoque \textbf{goloso} a la hora de diseñar este algoritmo, que comienza con una lista ordenada vacía $C$ de camiones (y por ende rutas) fabricando una solución vértice por vértice, es decir, constructivamente. El algoritmo toma como datos de entrada la lista de vértices $V$ que representan los \textit{clientes} que deben ser visitados por los camiones. Su funcionamiento está dividido en cuatro etapas definidas a continuación:

\begin{description}
\item[Paso 1.] Ordenar $V$ por la demanda de cada vértice
\item[Paso 2.] Elegir los vértices de mayor demanda compatibles con el espacio sobrante de mi último camión en $C$. De no existir ningún vértice con estas características, se despacha el camión al depósito, se invoca uno nuevo con capacidad libre máxima y se eligen los vértices más demandantes.
\item[Paso 3.] De la selección previa, se ordenan los vértices por cercanía al \textbf{depósito}
\item[Paso 4.] De estos últimos se escogen $K$ vértices y entre ellos se obtiene el nodo más cercano al camión en cuestión. Este será el siguiente vértice a ser visitado por un camión
\end{description}


% [√] 1. Bucket sort sobre capacidades de los vértices
% [√] 2. Para cada bucket, ordeno por cercanía al depósito
% [√] 3. Elijo el bucket de más capacidad compatible con el stock de mi camión
% [√] 4. Tomo los primeros k vértices de este bucket y me quedo con el más cercano al vértice donde estoy parado actualmente