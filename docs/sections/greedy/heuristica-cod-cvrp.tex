\begin{algorithm}[H]
\caption{\Comment $\mathcal{O}(????)$}
\begin{algorithmic}[1]
\Function{resolverCVRP}{Punto $deposito$, Conjunto de puntos $puntos$, Entero $capacidad$}
	\State \textbf{Buckets} $buckets \gets $\textbf{BucketSortPorDemanda}$(puntos)$ \Comment $\mathcal{O}(n + D)$
	\State \textbf{OrdenarCadaBucketPorCercaníaA}$(puntos, deposito)$ \Comment $\mathcal{O}(n*log(n))$
	\Statex
	\State \textbf{Entero} $vertices\_cubiertos \gets 0$
	\State \textbf{Lista de Camiones} $camiones \gets \varnothing$
	\State Insertar en $camiones$ un camión con capacidad $capacidad$ que empieza en $deposito$
	\Statex
	\While{$vertices\_cubiertos < |puntos|$}
		\State \textbf{Bucket} $bucket\_mas\_apto \gets$ \par
        \Statex[3] \textbf{EncontrarBucketMasApto}$(buckets, camiones, deposito, capacidad)$
		\State \textbf{Punto} $siguiente \gets$ \textbf{ExtraerVerticeMasApto}$(bucket\_mas\_apto, camiones)$
		\State Tomar el último camión agregado a $camiones$ y visitar el vértice $siguiente$
		\Statex
		\State $vertices\_cubiertos \gets vertices\_cubiertos + 1$
	\EndWhile

	\State \Return $camiones$
\EndFunction
\end{algorithmic}
\end{algorithm}

\vspace{10 pt}
\fbox{\parbox{0.9\textwidth}{
	Es necesario aclarar que el tipo de datos \textbf{Camión} lleva la cuenta de su stock disponible en forma de \textbf{Entero positivo} y los vértices que visita en un conjunto ordenado de \textbf{Lista de Puntos}. Además, llamamos un \textbf{Bucket} a una \textbf{Lista de Puntos} que comparten la misma demanda. Análogamente, llamamos \textbf{Buckets} a una \textbf{Lista de Buckets}, es decir, Buckets es un alias para una lista de lista de puntos. Hacemos este renombre para resaltar la diferencia entre una lista de puntos con la misma demanda (un bucket) y una lista de puntos cualquiera. El tipo de datos \textbf{Buckets} es obtenido a partir de la invocación de un \textbf{Bucket Sort} sobre una lista de puntos utilizando la demanda de cada vértice como valor comparativo.

	\vskip 8pt

	Como último, el valor $D$ mencionado en la complejidad temporal del \textbf{BucketSort} es la máxima demanda de un punto en la lista $puntos$.
}}
\vspace{10 pt}

\vskip 8pt