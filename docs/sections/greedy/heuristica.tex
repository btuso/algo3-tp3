\subsection{Heurística Constructiva Golosa}
\subsubsection{Diseño}
Utilizamos un enfoque \textbf{goloso} a la hora de diseñar este algoritmo, que comienza con una lista ordenada vacía $C$ de camiones (y por ende rutas) fabricando una solución vértice por vértice, es decir, constructivamente. El algoritmo toma como datos de entrada la lista de vértices $V$ que representan los \textit{clientes} que deben ser visitados por los camiones. Su funcionamiento está dividido en cuatro etapas definidas a continuación:

\begin{description}
\item[Paso 1.] Ordenar $V$ por la demanda de cada vértice
\item[Paso 2.] Elegir los vértices de mayor demanda compatibles con el stock disponible de mi último camión en $C$. De no existir ningún vértice con estas características, se despacha el camión al depósito, se invoca uno nuevo con stock lleno y se eligen los vértices más demandantes.
\item[Paso 3.] De la selección previa, se ordenan los vértices por cercanía al \textbf{depósito}
\item[Paso 4.] De estos últimos se escogen $K$ vértices y entre ellos se obtiene el nodo más cercano al camión en cuestión. Este será el siguiente vértice a ser visitado por un camión
\end{description}

\subsubsection{Pseudo-código}
El diseño de este algoritmo fue concebido con el objetivo de minimizar las distancias recorridas por los camiones al intentar vaciarlos lo más rápidamente debido a la temprana atención de los vértices más demandantes. Además de buscar una distancia mínima, con este criterio y bajo ciertas circunstancias también es posible observar una utilización de camiones cercana a la óptima. Pondremos a prueba estos casos más adelante. A continuación, exhibimos el pseudo-código del algoritmo:

\begin{algorithm}[H]
\caption{\Comment $\mathcal{O}(????)$}
\begin{algorithmic}[1]
\Function{resolverCVRP}{Punto $deposito$, Conjunto de puntos $puntos$, Entero $capacidad$}
	\State \textbf{Buckets} $buckets \gets $\textbf{BucketSortPorDemanda}$(puntos)$ \Comment $\mathcal{O}(n + D)$
	\State \textbf{OrdenarCadaBucketPorCercaníaA}$(puntos, deposito)$ \Comment $\mathcal{O}(n*log(n))$
	\Statex
	\State \textbf{Entero} $vertices\_cubiertos \gets 0$
	\State \textbf{Lista de Camiones} $camiones \gets \varnothing$
	\State Insertar en $camiones$ un camión con capacidad $capacidad$ que empieza en $deposito$
	\Statex
	\While{$vertices\_cubiertos < |puntos|$}
		\State \textbf{Bucket} $bucket\_mas\_apto \gets$ \par
        \Statex[3] \textbf{EncontrarBucketMasApto}$(buckets, camiones, deposito, capacidad)$
		\State \textbf{Punto} $siguiente \gets$ \textbf{ExtraerVerticeMasApto}$(bucket\_mas\_apto, camiones)$
		\State Tomar el último camión agregado a $camiones$ y visitar el vértice $siguiente$
		\Statex
		\State $vertices\_cubiertos \gets vertices\_cubiertos + 1$
	\EndWhile

	\State \Return $camiones$
\EndFunction
\end{algorithmic}
\end{algorithm}

\vspace{10 pt}
\fbox{\parbox{0.9\textwidth}{
	Es necesario aclarar que el tipo de datos \textbf{Camión} lleva la cuenta de su stock disponible en forma de \textbf{Entero positivo} y los vértices que visita en un conjunto ordenado de \textbf{Lista de Puntos}. Además, llamamos un \textbf{Bucket} a una \textbf{Lista de Puntos} que comparten la misma demanda. Análogamente, llamamos \textbf{Buckets} a una \textbf{Lista de Buckets}, es decir, Buckets es un alias para una lista de lista de puntos. Hacemos este renombre para resaltar la diferencia entre una lista de puntos con la misma demanda (un bucket) y una lista de puntos cualquiera. El tipo de datos \textbf{Buckets} es obtenido a partir de la invocación de un \textbf{Bucket Sort} sobre una lista de puntos utilizando la demanda de cada vértice como valor comparativo.

	\vskip 8pt

	Como último, el valor $D$ mencionado en la complejidad temporal del \textbf{BucketSort} es la máxima demanda de un punto en la lista $puntos$.
}}
\vspace{10 pt}

\vskip 8pt
Esta función representa la globalidad de la heurística. Comienza realizando un \textit{bucket sort} sobre la demanda de los puntos recibidos por parámetro y luego ordena cada bucket por cercanía al depósito con la función \textbf{OrdenarCadaBucketPorCercaníaA} en la línea 3. La intención de esta heurística es alejarse lo menos posible del depósito, priorizando los vértices más demandantes. Luego, se crea el primer camión y se lo agrega a la respectiva lista. Este camión – como todos – comenzará en el depósito. Luego, el bucle \textit{while} que le sigue intentará encontrar un vértice de demanda máxima que pueda ser satisfecho por el camión. De no encontrarse ninguno, devolverá el camión al depósito e invocará uno nuevo con stock máximo. Esta lógica es delegada en la función \textbf{ExtraerVérticeMásApto}, que devuelve el nodo más demandante posible, invocando o no un nuevo camión si es necesario. Luego, el vértice en cuestión es visitado por el camión y la cantidad de vértices cubiertos es incrementada en uno. El algoritmo se ejecutará $n$ veces, es decir, una vez por cada vértice pues cada iteración visita siempre uno.

\vskip 8pt

\begin{algorithm}[H]
\caption{\Comment $\mathcal{O}(n * log(n))$}
\begin{algorithmic}[1]
\Function{OrdenarCadaBucketPorCercaníaA}{Buckets $buckets$, Punto $punto$}
	\For{\textbf{Bucket} $bucket$ en $buckets$} \Comment $\mathcal{O}(D)$
		\State Ordenar los puntos de $bucket$ por distancia a $punto$ \Comment $\mathcal{O}(|bucket| * log(|bucket|))$
	\EndFor
	\Statex
	\State \Return $buckets$
\EndFunction
\end{algorithmic}
\end{algorithm}

Este algoritmo toma \textbf{Buckets} como parámetro de entrada y, para cada uno de ellos, ordena sus puntos por la distancia al vértice $punto$ adjuntado como dato de entrada. En el contexto general, esta función es utilizada para ordenar todos los vértices de todos los \textit{buckets} por cercanía al depósito del grafo. La complejidad de la estructura de iteración es $\mathcal{O}(D)$ porque dentro de $buckets$ hay $D$ listas de puntos, siendo $D$ la demanda más grande en la lista de puntos del grafo. Luego, la complejidad del sorting de cada bucket depende de la cantidad de elementos dentro de cada uno, es decir: $\mathcal{O}(|bucket| * log(|bucket|))$. Pero es fácil concluir que la complejidad total de esta función es $\mathcal{O}(n * log(n))$ porque a fin de cuentas se están ordenando $n$ vértices en total.

\begin{algorithm}[H]
\caption{\Comment $\mathcal{O}(D)$}
\begin{algorithmic}[1]
\Function{EncontrarBucketMásApto}{Buckets $buckets$, Lista de camiones $camiones$}
	\State \textbf{Entero} $stock\_restante \gets$ stock restante del último camión en $camiones$
	\Statex
	\State \textbf{Entero} $i \gets $\textbf{min}$(stock\_restante, |buckets| - 1)$
	\While{$i \geq 0 \land buckets[i] = \varnothing$}
		\State $i \gets i - 1$
	\EndWhile
	\Statex
	\If{$i < 0$}
		\State Enviar el último camión en $camiones$ al depósito
		\State Creo un nuevo camión en $camiones$
		\Statex
		\State $i \gets |buckets| - 1$
		\While{$buckets[i] = \varnothing$}
			\State $i \gets i - 1$
		\EndWhile
	\EndIf
	\Statex
	\State \Return $buckets[i]$
\EndFunction
\end{algorithmic}
\end{algorithm}

\vspace{10 pt}
\fbox{\parbox{0.9\textwidth}{
	Aclaración: todos los camiones creados inician su ruta en el depósito y con stock lleno. La dimensión de este stock es definida globalmente y es equivalente entre todos los camiones.
}}
\vspace{10 pt}

\vskip 8pt

Esta función se encarga de encontrar el bucket de mayor demanda (pues todos los vértices dentro de este bucket la comparten) tal que pueda ser satisfecha por un camión ya existente o si debe crearse uno nuevo. Su complejidad es $\mathcal{O}(D)$ porque en el peor de los casos, el último camión de la lista no tiene stock suficiente para ningún vértice en el grafo que no haya sido visitado. El primer bucle de la línea 4 comenzará explorando los \textit{buckets} desde el índice $stock\_restante$ hasta llegar a $0$, donde deberá detenerse y concluirá que no hay ningún vértice que cumpla que $demanda(v) \leq stock\_restante$ con $v$ algún vértice en los \textit{buckets} $[0, stock\_restante]$. Luego, se deberá crear un nuevo camión con stock máximo y se volverá a indagar la lista de \textit{buckets}. Sabemos que no hay vértices con demanda en los rangos $[0, stock\_restante]$ pues ya las iteramos en el bucle de la línea 4, pero sabemos que sí quedan vértices sin visitar en el grafo pues sino la función nunca se hubiese ejecutado. Ahora, como por enunciado sabemos que no existe un vértice con una demanda mayor al stock máximo de un camión, sabemos que hay un nodo sin visitar en el rango de índices $[stock\_restante + 1, D]$. En el peor de los casos, el vértice que buscamos está en el índice $stock\_restante + 1$, habiéndose concretado $D$ iteraciones. Finalmente, se retorna el \textit{bucket} en cuestión.

\begin{algorithm}[H]
\caption{\Comment $\mathcal{O}(K*log(K))$}
\begin{algorithmic}[1]
\Function{ExtraerVérticeMásApto}{Bucket $bucket$, Lista de camiones $camiones$, Entero $K$}
	\State \textbf{Punto} $actual \gets$ el último punto visitado por el último camión en $camiones$
	\State Ordenar los últimos $K$ puntos de $bucket$ de más lejano a más cercano a $actual$ \Comment $\mathcal{O}(K*log(K))$
	\Statex
	\State \textbf{Punto} $siguiente \gets$ último elemento de $bucket$ \Comment $\mathcal{O}(1)$
	\State $bucket \gets bucket - \{siguiente\}$ \Comment $\mathcal{O}(1)$
	\Statex
	\State \Return $siguiente$
\EndFunction
\end{algorithmic}
\end{algorithm}

El algoritmo superior recibe un \textit{bucket}, una lista de camiones y un entero $K$. El pedazo de código comienza obteniendo el último vértice visitado por el camión cuya ruta está siendo construída. En base a la posición de este nodo, se ordenan los últimos $K$ vértices de la lista $bucket$ por distancia Euclidiana de más lejano a más cercano. Aquí debemos explicar dos cosas:
\begin{enumerate}[a)]
\item ¿Por qué se ordenan los últimos $K$ vértices de la lista y su completitud? La respuesta a esta pregunta es sencilla: es otra de nuestras \textbf{heurísticas}. No creemos conveniente ordenar todo el \textit{bucket} por cada vértice, por lo tanto lo hacemos con los últimos $K$ ejemplares.
\item ¿Y por qué estos últimos $K$ vértices se ordenan de más lejano a más cercano? Porque sabemos que una vez sea elegido el nodo a visitar, este será removido. Como estamos trabajando sobre un vector, la complejidad de su remoción depende de cuántos elementos le siguen. Como es el último elemento, la complejidad de su remoción es constante.
\end{enumerate}

\subsubsection{Conclusiones}
Podemos entender que como $K$ es un dato heurístico constante, en general pequeño y además menor o igual a $n$ (pues al revés no tendría sentido), su efecto en la complejidad es absorbido por el crecimiento asintótico de $n$, por lo tanto la complejidad queda inafectada por $K$.

$$resolverCVRP = \mathcal{O}(n+D + n*log(n) + n*D*K*log(K))$$
$$= \mathcal{O}(n+D + n*log(n) + n*D)$$
$$= \mathcal{O}(n*log(n) + n*D)$$
$$= \mathcal{O}\Big( n*(log(n) + D)\Big)$$

\subsubsection{Instancias con soluciones no óptimas}
Algunas de las instancias del problema de Enrutamiento de Vehículos con Capacidad pueden brindar una solución muy distante de la óptima utilizando este algoritmo debido a las heurísticas que se utilizan. A continuación mencionaremos algunas:
\subsubsubsection{Camiones que ``van y vienen''}
La primera decisión del algoritmo es ordenar los vértices por demanda y luego cada \textit{bucket} por cercanía al depósito para finalmente obtener un \textit{bucket} de vértices satisfacibles por el stock de un camión. Consideremos este \textit{bucket} $B$ (que recordamos es una lista de vértices). Es posible que los $K$ elementos de $B$ más cercanos al depósito no estén cerca del último nodo $v$ visitado por el camión actual, sino por el contrario; estén en esquinas opuestas. Luego, al ordenar estos $K$ elementos por la distancia a $v$, los camiones recorrerán distancias muy largas con total de cumplir la heurística de satisfacer primero las demandas más altas.

\subsubsubsection{Demandas equivalentes}
Otro caso inoportuno para el algoritmo podría presentarse en forma de demandas equivalentes, donde ordenar por demanda no ofrezca absolutamente ninguna ventaja pues todas son iguales. Además, los vértices se terminarían ordenando por cercanía al depósito donde pueden ocurrir problemas como el de ``van y vienen'' explicado en el punto anterior.

\subsubsubsection{Demandas incongruentes con el stock de los camiones}
Al mismo tiempo, puede suceder que las elecciones de demandas no haga un uso óptimo del stock del camión. Imaginemos que tenemos un conjunto ordenado de vértices $V=[v_{1}, v_{2}, v_{3}, v_{4}]$ con demandas $D=[55, 50, 25, 25]$ y un stock máximo de $100$. Nuestro algoritmo resolvería visitar primero $v_{1}$ con demanda $55$, luego intentaría $v_{2}$ con demanda $50$ pero fallaría y finalmente – con $100 - 55 = 45$ de stock disponible – visitaría $v_{3}$ con una demanda de $25$, dejando un sobrante de stock de $20$ unidades insatisfacibles para el resto de los vértices visitados. En cambio, si el algoritmo no hubiese decidido visitar el nodo más demandante primero y hubiese resuelto visitar $v_{2}$, $v_{3}$ y $v_{4}$ con una demanda total de $100$, no nos sobraría stock y habríamos cubierto tres vértices con un solo camión (en vez de dos como en el caso anterior). El algoritmo no explora posibles soluciones vecinales por lo tanto, en algunos casos, se arriesga a ser ineficiente en la cantidad de camiones y en consecuencia en las distancias recorridas.

\subsubsubsection{Demanda máxima demasiado grande (limitación de diseño)}
Por una decisión de diseño, se decidió utilizar un \textit{bucket sort} para ordenar las demandas de los vértices en $\mathcal{O}(n)$. Si bien podríamos haber utilizado otro algoritmo en el orden de complejidad $\mathcal{O}(n*log(n))$, \textit{bucket sort} nos proveyó un entorno más cómodo y elegante y debido a que el universo de instancias con las cuales experimentaríamos disponía de un valor de demanda máxima acotada, creímos conveniente optar por esta opción. Sin embargo, podrían existir instancias donde $D$ sea extremadamente grande. Estas afectarían gravemente el consumo de memoria de la computadora y podrían inhabilitar el funcionamiento del algoritmo. En estos data-sets no es el caso.

% [√] 1. Bucket sort sobre capacidades de los vértices
% [√] 2. Para cada bucket, ordeno por cercanía al depósito
% [√] 3. Elijo el bucket de más capacidad compatible con el stock de mi camión
% [√] 4. Tomo los primeros k vértices de este bucket y me quedo con el más cercano al vértice donde estoy parado actualmente