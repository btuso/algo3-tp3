\subsubsection{Aclaraciones}

Para la siguiente experimentación, se optó por tomar dos casos distintos para graficar la performance de cada heurística:

\begin{itemize}
\item Medición de performance en base a tamaño del grafo.
\item Medición de performance en base a distribución del grafo.
\end{itemize}

\subsubsubsection{Medición en base a tamaño del grafo}
Para realizar el siguiente experimento, se tomaron datasets con distintas distribuciones del mismo tamaño y se tomó el promedio de ejecución para cada uno de ellos en base a 50 ejecuciones. En base a los resultados obtenidos, se seleccionó el mejor caso y se realizaron pruebas sobre datasets incrementando el tamaño.

\vskip 8pt

El punto de partida es un dataset de tamaño n = 45 ya que dicho tamaño se encontraba disponible para todos los tipos de datasets a experimentar.

\subsubsubsection{Medición en base a tamaño del grafo}
En base al mejor caso encontrado para el punto anterior, se utilizaron las demás instancias para mostrar como la distribución del grafo afecta la performance del mismo.
