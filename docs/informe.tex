% ******************************************************** %
%              TEMPLATE DE INFORME ORGA2 v0.1              %
% ******************************************************** %
% ******************************************************** %
%                                                          %
% ALGUNOS PAQUETES REQUERIDOS (EN UBUNTU):                 %
% ========================================t
%                                                          %
% texlive-latex-base                                       %
% texlive-latex-recommended                                %
% texlive-fonts-recommended                                %
% texlive-latex-extra?                                     %
% texlive-lang-spanish (en ubuntu 13.10)                   %
% ******************************************************** %


\documentclass[a4paper]{article}
\usepackage[spanish]{babel}
\usepackage[utf8]{inputenc}
\usepackage{charter}   % tipografia
\usepackage[pdftex]{graphicx}
\usepackage{amsmath}

\graphicspath{{./images/}}
\usepackage{paralist} %itemize inline
\usepackage{multirow}
\usepackage[table,xcdraw]{xcolor}

\usepackage{amsthm}
\usepackage{underscore}
\usepackage{caratula}
\usepackage{url}
\usepackage{multicol}

\usepackage{algpseudocode}% Provides: Pseudocode syntax for algorithms
\usepackage{color} % para snipets de codigo coloreados
\usepackage{fancybox}  % para el sbox de los snipets de codigo
\usepackage[shortlabels]{enumitem} % para a) b) c) ... en enums
%% EL COSO PAL SUBSUBSUBSUBSUBSECTION JAJA NO EM ARREPIENTO DE NADA
\usepackage{titlesec}
\usepackage{hyperref}

\titleclass{\subsubsubsection}{straight}[\subsection]

\newcounter{subsubsubsection}[subsubsection]
\renewcommand\thesubsubsubsection{\thesubsubsection.\arabic{subsubsubsection}}
\renewcommand\theparagraph{\thesubsubsubsection.\arabic{paragraph}} % optional; useful if paragraphs are to be numbered

\titleformat{\subsubsubsection}
  {\normalfont\normalsize\bfseries}{\thesubsubsubsection}{1em}{}
\titlespacing*{\subsubsubsection}
{0pt}{3.25ex plus 1ex minus .2ex}{1.5ex plus .2ex}

\makeatletter
\renewcommand\paragraph{\@startsection{paragraph}{5}{\z@}%
  {3.25ex \@plus1ex \@minus.2ex}%
  {-1em}%
  {\normalfont\normalsize\bfseries}}
\renewcommand\subparagraph{\@startsection{subparagraph}{6}{\parindent}%
  {3.25ex \@plus1ex \@minus .2ex}%
  {-1em}%
  {\normalfont\normalsize\bfseries}}
\def\toclevel@subsubsubsection{4}
\def\toclevel@paragraph{5}
\def\toclevel@paragraph{6}
\def\l@subsubsubsection{\@dottedtocline{4}{7em}{4em}}
\def\l@paragraph{\@dottedtocline{5}{10em}{5em}}
\def\l@subparagraph{\@dottedtocline{6}{14em}{6em}}
\makeatother

\setcounter{secnumdepth}{4}
\setcounter{tocdepth}{4}
%% FIN

% ******* Paquetes agregados por mi
\usepackage{graphicx} % Provides: image embedding
\usepackage{amsmath} % Provides: begin{align}
\usepackage{ amssymb } % Provides: \varnothing
\usepackage{enumitem} % Provides: better enumerates
\usepackage{algorithm}% Provides: \begin{algorithm}
\usepackage{algpseudocode}% Provides: Pseudocode syntax for algorithms
\usepackage{float}% Provides: floating section anchoring
\usepackage[bottom]{footmisc} % Provides: sends footnotes to the bottom

% ********************************************************* %
% ~~~~~~~~              Code snippets             ~~~~~~~~~ %
% ********************************************************* %

\usepackage{color} % para snipets de codigo coloreados
\usepackage{fancybox}  % para el sbox de los snipets de codigo

\definecolor{litegrey}{gray}{0.94}
\definecolor{white}{RGB}{255,255,255}

\newenvironment{codesnippet}{%
	\begin{Sbox}\begin{minipage}{\textwidth}\sffamily\small}%
	{\end{minipage}\end{Sbox}%
		\begin{center}%
		\vspace{-0.4cm}\colorbox{litegrey}{\TheSbox}\end{center}\vspace{0.3cm}}



% ********************************************************* %
% ~~~~~~~~         Formato de las páginas         ~~~~~~~~~ %
% ********************************************************* %

\usepackage{fancyhdr}
\pagestyle{fancy}


\renewcommand{\sectionmark}[1]{\markright{\thesection\ - #1}}

\fancyhf{}

\fancyhead[LO]{Sección \rightmark} % \thesection\ 
\fancyfoot[LO]{\small{Bernardo Tuso, Celeste Rodriguez, Facundo Linlaud, Philip Garrett}}
\fancyfoot[RO]{\thepage}
\renewcommand{\headrulewidth}{0.5pt}
\renewcommand{\footrulewidth}{0.5pt}
\setlength{\hoffset}{-0.8in}
\setlength{\textwidth}{16cm}
\setlength{\headsep}{0.5cm}
\setlength{\textheight}{25cm}
\setlength{\voffset}{-0.7in}
\setlength{\headwidth}{\textwidth}
\setlength{\headheight}{13.1pt}

\renewcommand{\baselinestretch}{1.1}  % line spacing

\usepackage{xspace}
\newcommand{\Alpha}{\ensuremath{\mathsf{a}}\xspace}
\newcommand{\Red  }{\ensuremath{\mathsf{r}}\xspace}
\newcommand{\Green}{\ensuremath{\mathsf{g}}\xspace}
\newcommand{\Blue }{\ensuremath{\mathsf{b}}\xspace}

% ******************************************************** %

\begin{document}

\thispagestyle{empty}
\materia{Algoritmos y Estructuras de Datos III}
\submateria{Segundo Cuatrimestre de 2018}
\titulo{Trabajo Práctico III}
\subtitulo{Enrutamiento de Vehículos con Capacidad}
\integrante{Bernardo Tuso}{792/14}{btuso.95@gmail.com}
\integrante{Celeste Rodriguez}{639/16}{cmrodriguez997@gmail.com}
\integrante{Facundo Linlaud}{561/16}{facundolinlaud@gmail.com}
\integrante{Philip Garrett}{318/14}{garrett.phg@gmail.com}

\maketitle
\newpage
%_________________________________________________________________________________________________________________

\thispagestyle{empty}
\vspace{3cm}
\tableofcontents
\newpage

\section{Introducción}
\subsection{Enrutamiento de Vehículos con Capacidad ¿De qué hablamos?}
Supongamos que somos distribuidores de un producto, tenemos varios clientes en distintas ciudades a los cuales debemos hacer llegar nuestro producto con nuestro camión distribuidor. Ahora bien, queremos lograr esto recorriendo la menor distancia posible y visitando a cada cliente una única vez para luego volver a nuestro depósito, ¿cómo lograrlo? Podemos simplemente dibujar en el mapa y tardar un buen tiempo hasta resolverlo. Nuestro negocio crece, y ahora tenemos una flota de camiones disponible para realizar la tarea. Sin embargo, no queremos que dos camiones pasen por la misma ciudad/cliente, pero también queremos realizar la distribución en recorriendo la menor distancia, como antes. ¿Cómo resolver nuestra inquietud? ¿Cómo diseñar rutas que cumplan el objetivo?

\vskip 8pt

La problemática descrita anteriormente es conocida como ``Problema de Enrutamiento de Vehículos'', VRP por sus siglas en inglés. Se trata de un problema de optimización combinatoria en el que debemos cumplir ciertos requisitos:

\begin{itemize}
	\item Cada ruta que diseñemos para visitar un cliente, debe ser de costo mínimo
	\item Cada camión debe iniciar y finalizar su recorrido en el depósito
	\item Cada cliente solo puede ser visitado una única vez
	\item Cada cliente solo puede ser visitado por un vehículo (corolario del anterior)
\end{itemize}

\vskip 8pt

Nos encontramos con una restricción más, cada vehículo tiene una capacidad máxima, dado que somos distribuidores que cumplen la ley, no debemos sobrepasar dicha capacidad. Generamos entonces una variante del problema original, ante nosotros el problema de ``Enrutamiento de Vehículos con Capacidad'', también conocido como CVRP.

\vskip 8pt

En términos más formales, sea G un grafo cualquiera con V su conjunto de vértices y E su conjunto de aristas, que asumiremos no dirigidas. En V encontraremos un primer nodo representante de nuestro depósito y los nodos siguientes los clientes a visitar. La distancia, o costo, entre dos vértices es representada por un valor asociado a cada arista entre un par de vértices ($i$,$j$). Cada vértice del grafo cuenta con una indicación acerca del volumen que debemos entregar a cada cliente. El costo total de una ruta es la suma de los costos de las rutas que lo componen, cada cual tiene su costo compuesto por la suma de cada una de las aristas que lo componen.

\subsubsection{Del papel a la práctica}
Dejando de lado la teoría, ¿Cómo podemos llevar el problema de ``Enrutamiento de Vehículos con Capacidad'' a la vida real? ¿A qué situación podemos vincular un modelo de CVRP?

En primer lugar, continuando con el ejemplo presentado anteriormente, si nosotros tuvieramos que distribuir un producto para varios clientes, seguramente nos inclinaríamos por una solución que cumpla con este modelo. Según describe el paper ``The Impact of a Decision-Support System for Vehicle Routeing in a Foodservice Supply Situation'' de Steven R. Evans y John P. Norback, la empresa Kraft ha determinado aproximadamente un ahorro del 10,7\% en 10 casos de rutas de distribución.

Más cercano al día a día podemos encontrar el caso de los camiones recolectores de basura. Los mismos no distribuyen un producto si no que recolectan. Sin embargo, también cuentan con una capacidad limitada para cargar, parten de un depósito y vuelven al mismo y deben pasar por cada punto sin dejar basura no recolectada.

\subsubsection{Ejemplos}

\end{document}

